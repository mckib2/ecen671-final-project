\documentclass[journal]{IEEEtran}
\usepackage{cite}
\usepackage{amssymb}

\title{Optimizing Compressed Sensing Sensing Matrices}
\author{Nicholas McKibben and Connor Anderson}

\begin{document}
\maketitle
\thispagestyle{empty}
\pagestyle{empty}
%%%%%%%%%%%%%%%%%%%%%%%%%%%%%%%%%%%%%%%%%%%%%%%%%%%%%%%%%%%%%%%%%%%%%%%%%%%%%%%%
\begin{abstract}

An abstract

\end{abstract}
%%%%%%%%%%%%%%%%%%%%%%%%%%%%%%%%%%%%%%%%%%%%%%%%%%%%%%%%%%%%%%%%%%%%%%%%%%%%%%%%
\section{INTRODUCTION}

Compressed sensing is a technique for reconstructing a signal that has been
highly undersampled. It is impossible to recover an undersampled signal in 
general: there is no way to know what the missing information was. However, when
the original signal meets certain constraints, it becomes possible to recover it
from its undersampled representation with a high (or even perfect) degree of
accuracy. For example, if the highest-frequency component $f$ of a signal is
known, the signal can be perfectly recovered if it is sampled uniformly with a 
sampling rate of $2f$ (the Nyquist sampling theorem). Compressed sensing is a
method for recovering signals that have been undersampled to an even greater
degree. In order for a signal to be recoverable through compressed sensing, it
must meet two constraints. First, it must have a sparse representation in some
transform domain. Second, the sampling must be done incoherently so that the
resulting aliasing appears like a relatively low level of noise in the transform
domain. Luckily, many signals of interest can be shown to have sparse
representations is some domain, such as the Wavelet Transform or the Discrete
Cosine Transform. If those signals are then sampled in the right way, they can
be represented by far fewer data points but recovered almost exactly.

In this work, we explore compressed sensing for recovering highly undersampled
images. We outline the method and mathematics involved, as well as some recent
developments in more effective sensing matrices.

%%%%%%%%%%%%%%%%%%%%%%%%%%%%%%%%%%%%%%%%%%%%%%%%%%%%%%%%%%%%%%%%%%%%%%%%%%%%%%%%
\section{BODY}

Some body content. Here's a citation \cite{example}.

%%%%%%%%%%%%%%%%%%%%%%%%%%%%%%%%%%%%%%%%%%%%%%%%%%%%%%%%%%%%%%%%%%%%%%%%%%%%%%%%
\section{CONCLUSION}

A conclusion


%%%%%%%%%%%%%%%%%%%%%%%%%%%%%%%%%%%%%%%%%%%%%%%%%%%%%%%%%%%%%%%%%%%%%%%%%%%%%%%%
\bibliographystyle{IEEEtran}
\bibliography{project}{}
%%%%%%%%%%%%%%%%%%%%%%%%%%%%%%%%%%%%%%%%%%%%%%%%%%%%%%%%%%%%%%%%%%%%%%%%%%%%%%%%

\end{document}
